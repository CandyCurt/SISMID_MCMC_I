\documentclass[11pt]{article}

\setlength{\oddsidemargin}{0.1in}
\setlength{\textwidth}{6.5in}
\setlength{\topmargin}{-0.5in}
\setlength{\textheight}{9in}
\renewcommand{\baselinestretch}{1.3} 

\usepackage{fancyhdr}
%%\usepackage{fullpage}
%%\usepackage{subfigure}
\usepackage{latexsym}
\usepackage{wrapfig}
\usepackage{graphicx}
\usepackage{amsmath, amsthm, amssymb}
\usepackage{natbib}
\usepackage{algorithm}
\usepackage{algorithmic}
\numberwithin{algorithm}{section}
%%\usepackage{amsfonts}
%\usepackage{nopageno}

%%\usepackage{graphicx}        % standard LaTeX graphics tool
                             % when including figure files
%%\usepackage{multicol}        % used for the two-column index
%%\usepackage[bottom]{footmisc}% places footnotes at page bottom


%% my additional packages 
%%\usepackage{lscape}
%%\usepackage{fancybox}
%%\usepackage{amsfonts,amssymb,amsfonts,amsmath}
%%\usepackage{threeparttable}




%\newcommand{\draftnote}[1]{\marginpar{\tiny\raggedright\textsf{\hspace{0pt}{\bf Comments}:#1}}}
%\newcommand{\draftnote}[1]{}

%%\DeclareMathOperator*{\argmax}{arg\,max}

\newcommand{\cprob}[2]{\ensuremath{\text{Pr}\left(#1 \,|\,#2\right)}}  
\newcommand{\prob}[1]{\ensuremath{\text{Pr}\left(#1 \right)}}
\newcommand{\cexpect}[4]{\ensuremath{\text{E}\left#3 #1 \,|\,#2\right#4}}  
\newcommand{\expect}[3]{\ensuremath{\text{E}\left#2 #1 \right#3}}

\newcommand{\fder}[1]{\frac{d}{d #1}}
\newcommand{\hder}[2]{\frac{d^{#2}}{d {#1}^{#2}}}
\newcommand{\fpart}[1]{\frac{\partial}{\partial #1}}
\newcommand{\hpart}[2]{\frac{\partial^{#2}}{\partial {#1}^{#2}}}
\newcommand{\iid}{\ensuremath{\overset{\text{iid}}{\sim}}}
\newcommand{\indfun}[1]{\ensuremath{1_{\{#1\}}}}
\newcommand{\asarrow}{\ensuremath{\overset{\text{a.s.}}{\rightarrow}}}
\newcommand{\parrow}{\ensuremath{\overset{\text{P}}{\rightarrow}}}
\newcommand{\darrow}{\ensuremath{\overset{\text{D}}{\rightarrow}}}
\newcommand{\mydef}{\ensuremath{\overset{\text{def}}{=}}}

\DeclareMathOperator*{\argmax}{arg\,max}
\DeclareMathOperator*{\argmin}{arg\,min}

\newtheorem*{theorem}{Theorem}
\newtheorem*{prop}{Proposition}
\newtheorem*{corollary}{Corollary}
\newtheorem*{lemma}{Lemma}

\theoremstyle{remark}
\newtheorem*{mynote}{Note}

\theoremstyle{definition}
\newtheorem*{define}{Definition}

\newenvironment{example}[1]{\begin{trivlist}
\item[\hskip \labelsep {\bfseries Example}: \underline{#1}]\ \\}{\end{trivlist}}

\newenvironment{Proof}{\begin{trivlist}
\item[\hskip \labelsep \textit{Proof}:]}{\end{trivlist}}

\newenvironment{exercise}{\begin{trivlist}
\item[\hskip \labelsep \textit{Exercise}:]}{\end{trivlist}}

\newcommand{\todo}[1]{\begin{center}To do: {\bf #1}\end{center}}

%% group numberging of equations and figures by section

\numberwithin{equation}{section}
\numberwithin{figure}{section}

\bibliographystyle{plainnat}


\pagestyle{fancy}
\renewcommand{\headrulewidth}{0.8pt}
\lhead{\bf \large SISMID, Module 7 Practicals}
\rhead{\bf Summer 2015}
\chead{} 
\lfoot{} 
\rfoot{} 
\cfoot{}

\begin{document}

\setkeys{Gin}{width=1.0\textwidth}


\begin{center}
  \textbf{\Large Practical: Monte Carlo and Markov chain theory}\\
  {\large Instructors: Kari Auranen, Elizabeth Halloran and Vladimir Minin}\\
  {\large July 13 -- July 15, 2015}
\end{center}


\section*{Estimating the tail of the standard normal distribution}
Let $Z \sim \mathcal{N}(0,1)$. We would like to estimate the tail probability 
$\text{Pr}(Z > c)$, where $c$ is large (e.g. $c = 4.5$). 

\subsection*{Your task}
Implement naive and importance sampling Monte Carlo estimates of $\text{Pr}(Z>4.5)$, where
$Z \sim \mathcal{N}(0,1)$. Download `import\_sampl\_reduced.R' from 
the course web page. The code has a couple of things to get you started.

\section*{Ehrenfest model of diffusion}
  Consider the Ehrenfest model with $N=100$ gas molecules. From our derivations we know that 
  the stationary distribution of the chain is $\text{Bin}(\frac{1}{2},N)$. The chain is irreducible
  and positive recurrent (why?). The stationary variance can be computed analytically as 
  $N\times\frac{1}{2}\times\frac{1}{2}$. 

\subsection*{Your task}
Use ergodic theorem to approximate the stationary variance and compare your estimate with the analytical result.
Don't panic! You will not have to write everything from scratch. Download `ehrenfest\_diff\_reduced.R' file from the course
web page. Follow comments in this R script to fill gaps in the code. 




\end{document}

