\documentclass[11pt]{article}

\setlength{\oddsidemargin}{0.1in}
\setlength{\textwidth}{6.5in}
\setlength{\topmargin}{-0.5in}
\setlength{\textheight}{9in}
\renewcommand{\baselinestretch}{1.3} 

\usepackage{fancyhdr}
%%\usepackage{fullpage}
%%\usepackage{subfigure}
\usepackage{latexsym}
\usepackage{wrapfig}
\usepackage{graphicx}
\usepackage{amsmath, amsthm, amssymb}
\usepackage{natbib}
\usepackage{algorithm}
\usepackage{algorithmic}
\numberwithin{algorithm}{section}
%%\usepackage{amsfonts}
%\usepackage{nopageno}

%%\usepackage{graphicx}        % standard LaTeX graphics tool
                             % when including figure files
%%\usepackage{multicol}        % used for the two-column index
%%\usepackage[bottom]{footmisc}% places footnotes at page bottom


%% my additional packages 
%%\usepackage{lscape}
%%\usepackage{fancybox}
%%\usepackage{amsfonts,amssymb,amsfonts,amsmath}
%%\usepackage{threeparttable}




%\newcommand{\draftnote}[1]{\marginpar{\tiny\raggedright\textsf{\hspace{0pt}{\bf Comments}:#1}}}
%\newcommand{\draftnote}[1]{}

%%\DeclareMathOperator*{\argmax}{arg\,max}

\newcommand{\cprob}[2]{\ensuremath{\text{Pr}\left(#1 \,|\,#2\right)}}  
\newcommand{\prob}[1]{\ensuremath{\text{Pr}\left(#1 \right)}}
\newcommand{\cexpect}[4]{\ensuremath{\text{E}\left#3 #1 \,|\,#2\right#4}}  
\newcommand{\expect}[3]{\ensuremath{\text{E}\left#2 #1 \right#3}}

\newcommand{\fder}[1]{\frac{d}{d #1}}
\newcommand{\hder}[2]{\frac{d^{#2}}{d {#1}^{#2}}}
\newcommand{\fpart}[1]{\frac{\partial}{\partial #1}}
\newcommand{\hpart}[2]{\frac{\partial^{#2}}{\partial {#1}^{#2}}}
\newcommand{\iid}{\ensuremath{\overset{\text{iid}}{\sim}}}
\newcommand{\indfun}[1]{\ensuremath{1_{\{#1\}}}}
\newcommand{\asarrow}{\ensuremath{\overset{\text{a.s.}}{\rightarrow}}}
\newcommand{\parrow}{\ensuremath{\overset{\text{P}}{\rightarrow}}}
\newcommand{\darrow}{\ensuremath{\overset{\text{D}}{\rightarrow}}}
\newcommand{\mydef}{\ensuremath{\overset{\text{def}}{=}}}

\DeclareMathOperator*{\argmax}{arg\,max}
\DeclareMathOperator*{\argmin}{arg\,min}

\newtheorem*{theorem}{Theorem}
\newtheorem*{prop}{Proposition}
\newtheorem*{corollary}{Corollary}
\newtheorem*{lemma}{Lemma}

\theoremstyle{remark}
\newtheorem*{mynote}{Note}

\theoremstyle{definition}
\newtheorem*{define}{Definition}

\newenvironment{example}[1]{\begin{trivlist}
\item[\hskip \labelsep {\bfseries Example}: \underline{#1}]\ \\}{\end{trivlist}}

\newenvironment{Proof}{\begin{trivlist}
\item[\hskip \labelsep \textit{Proof}:]}{\end{trivlist}}

\newenvironment{exercise}{\begin{trivlist}
\item[\hskip \labelsep \textit{Exercise}:]}{\end{trivlist}}

\newcommand{\todo}[1]{\begin{center}To do: {\bf #1}\end{center}}

%% group numberging of equations and figures by section

\numberwithin{equation}{section}
\numberwithin{figure}{section}

\bibliographystyle{plainnat}


\pagestyle{fancy}
\renewcommand{\headrulewidth}{0.8pt}
\lhead{\bf \large SISMID, Module 8 Practicals}
\rhead{\bf Summer 2017}
\chead{} 
\lfoot{} 
\rfoot{} 
\cfoot{}

\begin{document}

\setkeys{Gin}{width=1.0\textwidth}


\begin{center}
  \textbf{\Large Practical: Metropolis-Hastings Algorithm}\\
  {\large Instructors: Kari Auranen, Elizabeth Halloran and Vladimir Minin}\\
  {\large July 17 -- July 19, 2017}
\end{center}

\section*{Sampling from the standard normal distribution}
Suppose our target is a univariate standard normal distribution with density $f(x) = 1/(\sqrt{2\pi})e^{-x^2/2}$.
Given current state $x^{(t)}$, we generate two uniform r.v.s $U_1 \sim U[-\delta,\delta]$ and 
$U_2 \sim U[0,1]$. Then set
\[
x^{(t+1)} = 
\begin{cases}
  x^{(t)} + U_1 &\text{ if } U_2 \le \min\left\{e^{\left[\left(x^{(t)}\right)^2 - (x^{(t)}+U_1)^2\right]/2},1\right\} \\
  x^{(t)} & \text{ otherwise}.
\end{cases}
\]
$\delta$ is a tuning parameter. Large $\delta$ leads to small acceptance rate, small $\delta$ leads to slow
exploration of the state space. The rule of thumb for random walk proposals is to keep acceptance probabilities
around 30-40\%. If your proposal is close to the target, then higher acceptance rates are favorable.


\subsection*{Your task}
Implement the above algorithm. Experiment with the tuning parameter $\delta$ and report empirically estimate
acceptance probabilities for different values of this parameter.

\section*{Distribution of the time of infection}
Consider a two state continuous-time Markov SIS model, where the disease status $X_t$ cycles between the two states: 
$1$=susceptible, $2$=infected. Let the infection rate be $\lambda_1$ and clearance rate be $\lambda_2$.
Suppose that an individual is susceptible at time $0$ ($X_0 = 1$) and infected at time $T$ ($X_T=2$). We don't know
anything else about the disease status of this individual during the interval $[0,T]$. If $T$ is small enough, 
it is reasonable to assume that the individual was infected only once during this time interval. We would like 
to obtain the distribution of the time of infection $I$, conditional on the information we have.

\subsection*{Your task}
Implement a Metropolis-Hastings sampler to draw realizations from the distribution 
\[
\text{Pr}(I \mid X_0=1, X_t=2, N_t=1) \propto \text{Pr}(0<t<I: X_t=1, I<t<T: X_t=2),
\] 
where $N_t$ is the number of infections. Since $X_t$ is a continuous-time Markov chain, the last 
probability (it is actually a density) can be written as
\[
\text{Pr}(0<t<I: X_t=1, I<t<T: X_t=2) = \underbrace{\lambda_1 e^{-\lambda_1 I}}_{\text{density of waiting time until infection}} 
\times \overbrace{e^{-\lambda_2 (T-I)}}^{\text{prob of staying infected}}.
\]
To make things concrete, set $\lambda_1 = 0.1$, $\lambda_2=0.2$ and $T=1.0$. For your proposal distribution, use 
a uniform random walk with reflective boundaries $0$ and $T$. In other words, given a current value of the infection
time $t_c$, generate $u = \text{Unif}_{[t_c-\delta,t_c+\delta]}$ ($2\delta < T$) and then make a proposal value
\[
t_p = 
\begin{cases}
  u &\text{ if } 0 < u < T,\\
  2T - u &\text{ if } u > T,\\
  -u &\text{ if } u < 0.
\end{cases}
\]
This is a symmetric proposal, so your M-H ratio will contain only the ratio of target densities:
\[
\frac{\lambda_1 e^{-\lambda_1 t_p}e^{-\lambda_2 (T-t_p)}}{\lambda_1 e^{-\lambda_1 t_c}e^{-\lambda_2 (T-t_c)}} = 
e^{-\lambda_1(t_p - t_c) -\lambda_2(t_c-t_p)} = e^{(t_p-t_c)(\lambda_2-\lambda_1)}.
\]
Plot the histogram of the posterior distribution of the infection time. Try a couple of sets of values for $\lambda_1$ and $\lambda_2$ and examine
the effect of these changes on the posterior distribution of the infection time.

\end{document}

